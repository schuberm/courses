\documentclass{article}

\usepackage{amsmath,graphicx,parskip}
\usepackage{fancyhdr}
\usepackage[english]{babel}
\usepackage{verbatim}
\usepackage[top=3cm,bottom=3cm]{geometry}
\pagestyle{fancy}
\lhead{Samuel Cole Huberman}
\chead{MIE1011: Problem Set 8}
\rhead{999157923}

\begin{document}

\section*{Chapter 2 Question 4}
\subsection*{Part A}
The Helmholtz function for the surface is defined as
\begin{align*}
F&=U-TS \\
&=T^{LV}S^{LV}+\gamma^{LV} A^{LV} + \sum_i \mu_i^{LV} N_i^{LV}-T^{LV}S^{LV}
\end{align*}
Dividing through by the area
\begin{align*}
f=\gamma^{LV} +\sum_i \mu_i^{LV} N_i^{LV}
\end{align*}
Since $\gamma^{LV}=\gamma^{LV}(T)$, f is function of and $T$ and $\mu_i^{LV}$.

\subsection*{Part B}
Constructing the Helmholtz function of the composite system
\begin{align*}
F^C&=U^C-T^CS^C\\
&=U^L+U^V+U^{LV}+U^R-T^R(S^L+S^V+S^{LV}+S^R)
\end{align*}
From the energy and entropy postulates
\begin{align*}
0&=\Delta(U^L+U^V+U^{LV}+U^R)\\
0&\ge \Delta(S^L+S^V+S^{LV}+S^R)
\end{align*}
Any arbitrary change would increase $F^{C}$, so at equilibrium, $F^{C}$ must be at a minimum.

\subsection*{Part C}
Taking virtual displacements of $F^C$
\begin{align*}
dF^C=\sum(\mu^V-\mu^L)dN^V+\sum(\mu^{LV}-\mu^L)dN^{LV}+(-4\pi R^2P^V+4\pi R^2P^L+8\pi\gamma^{LV}R)dR
\end{align*}
Thus the constraints for equilibrium are
\begin{align*}
\mu^V&=\mu^L=\mu^{LV}\\
P^V &=P^L +\frac{2\gamma^{LV}}{R}
\end{align*}

\subsection*{Part D}
Following the notes, we take the derivative of the difference between real and virtual states\begin{align*}
\frac{dB(R,R_{\epsilon})-B_0}{dR}&=4\pi\gamma^{LV}(2R-\frac{3R^2}{R_{\epsilon}})=0\\
R&=\frac{2R_{\epsilon}}{3}
\end{align*}



\end{document}
