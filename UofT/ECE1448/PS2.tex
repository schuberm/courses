\documentclass{article}

\usepackage{amsmath,graphicx,parskip}
\usepackage{fancyhdr}
\usepackage[english]{babel}
\pagestyle{fancy}
\lhead{Samuel Cole Huberman}
\chead{ECE1448: Problem Set 2}
\rhead{999157923}

\newcommand{\unit}[1]{\ensuremath{\, \mathrm{#1}}}
\numberwithin{equation}{section}

\begin{document}
\large
\section*{Particle in a Spherical Box}
For a particle enclosed in a spherical box, the potential is piecewise:
\begin{center}
$$
V(r) =
\begin{cases}
0 & 0 \leq r\leq a=2 \AA \\
\infty & otherwise\\
\end{cases}
$$
\end{center}
To determine the possible energies, $W$, of the system, we must first solve the 3D Schrodinger equation for a particle in a spherical box:
\begin{align*}
	-\frac{\hbar^2}{2m_0}\nabla^2\Psi &= W\Psi
\end{align*}
Where $\nabla^2$ is the Laplacian operator in spherical coordinates and $m_0$ is the mass of the particle. Relying upon seperation of variables to determine the form of the solution, we substitute $Y^m_l(\theta,\phi)R_{n,l}(r)$. In this case, the angular components of the wavefunction are equivalent to the spherical harmonics from the hydrogen atom:
\begin{align*}
	Y^m_l(\theta,\phi) &= P^m_l(\cos\theta)e^{\pm im\phi}\\
	P^m_l(x) &=(1-x^2)^{|m|/2}\frac{d}{dx}^{|m|}P_l(x)\\
        P_l(x)&=\frac{1}{2^ll!}(\frac{d}{dx})^l(x^2-1)^l
\end{align*}
Inside the sphere, the separated radial equation is, where $k=\frac{\sqrt{2mW}}{\hbar}$:
\begin{align*}
	\frac{d}{dr}(r^2\frac{dR_{n,l}}{dr})+\frac{2mr^2}{\hbar}W_{n,l}R&=l(l+1)R_{n,l}
%	\frac{2}{r}\frac{dR_{n,l}}{dr}+\frac{d^2R_{n,l}}{dr^2}+(k^2-\frac{l(l+1)}{r^2})R_{n,l}&=0
\end{align*}
If we let $x=kr$ so $dx=kdr$, the equation becomes:
\begin{align*}
        \frac{d}{dx}(x^2\frac{dR_{n,l}}{dx})+(x^2-l(l+1))R_{n,l}=0
\end{align*}
We know the solution takes the form:
\begin{align*}
	R(r)&=\frac{J_{l+1/2}(kr)}{\sqrt{kr}}
\end{align*}
Where $J_{n+1/2}$ is the Bessel function of half integral order. At the boundary, $R(a)=0$, so:
\begin{align*}
	\frac{J_{l+1/2}(ka)}{\sqrt{ka}}&=0\\
	ka&=z_{n,l}
\end{align*}
Where $z_{n,l}$ is the nth zero of the Bessel function. Rearranging for W gives:
\begin{align*}
	W_{n,l}&=z_{n,l}^2\frac{\hbar^2}{2m_0a^2}
\end{align*}
Clearly, the energy does not depend on the magnetic quantum number $m$, so for every value of the azimuthal quantum number $l$ we can have $m=-l..0..l$ that will yield the same energy. Therefore each energy level is $2l+1$-fold degenerate.
\begin{table}
 \begin{center}
  \begin{tabular}{| l |l |l |l |l |}
  \hline
  $n,l$ & $zero$ &Degeneracy& $Energy (\frac{\hbar^2}{2m_0a^2})$ &$Energy (eV)$\\
  \hline
  n=1, l=0 & 3.142& 1  &$3.142^2$ &$3.142^2*0.9524=9.402$\\ \hline
  n=1, l=1 & 4.493& 3  &$4.493^2$ &$4.493^2*0.9524=19.226$\\ \hline
  n=1, l=2 & 5.763& 7  &$5.763^2$ &$5.763^2*0.9524=31.631$\\ \hline
  n=2, l=0 & 6.283& 1  &$6.283^2$ &$6.283^2*0.9524=37.597$\\ \hline
  \end{tabular}
  \caption{Table 2: Four lowest energies for spherical box}
 \end{center}
\end{table}

\section*{Particle in a Cubic Box}
For a particle enclosed in a spherical box, the potential is piecewise ($a=2\AA$):
\begin{center}
$$
V(r) =
\begin{cases}
0 & 0 <x <a\sqrt[3]{\frac{4\pi}{3}}, 0 <y <a\sqrt[3]{\frac{4\pi}{3}},0 <z <a\sqrt[3]{\frac{4\pi}{3}}, \\
\infty & otherwise\\
\end{cases}
$$
\end{center}
The 3D Schrodinger equation for a particle in a cubic box (for simplification, let $b=a\sqrt[3]{\frac{4\pi}{3}}$):
\begin{align*}
	-\frac{\hbar^2}{2m}\nabla^2\Psi &= W\Psi
\end{align*}
Where $\nabla^2$ is the Laplacian operator in Cartesian coordinates. Using separation of variables [$\Psi(x,y,z)=X(x)Y(y)Z(z)$] to solve the 3D Schrodinger Equation for a particle in a (cubic) box, we find:
\begin{align*}
	-\frac{\hbar^2}{2m}\frac{X''}{X} &= W_1
\\	-\frac{\hbar^2}{2m}\frac{Y''}{Y} &= W_2
\\	-\frac{\hbar^2}{2m}\frac{Z''}{Z} &= W_3
\end{align*}
Where each equation is equivalent to that describing a 1D particle in the box with the same boundary conditions ($[X(0),Y(0),Z(0)]=0$ and $[X(b),Y(b),Z(b)]=0$) Therefore, we can write:
\begin{align*}
	X(x)_{n}=\sqrt{\frac{1}{2b}}\sin (\frac{n_x \pi}{b}x)
\\	Y(y)_{n}=\sqrt{\frac{1}{2b}}\sin (\frac{n_y \pi}{b}y)
\\	Z(z)_{n}=\sqrt{\frac{1}{2b}}\sin (\frac{n_z \pi}{b}z)
\end{align*}
The overall wavefunction is then:
\begin{align*}
	\Psi(x,y,z)_{n}=\sqrt{\frac{8}{b^3}}\sin (\frac{n_x \pi}{b}x)\sin (\frac{n_y \pi}{b}y)\sin (\frac{n_z \pi}{b}z)
\end{align*}
The energies are:
\begin{align*}
	W_{1n}=\frac{n_x^2\pi^2\hbar^2}{2mb^2}
\\	W_{2n}=\frac{n_y^2\pi^2\hbar^2}{2mb^2}
\\	W_{3n}=\frac{n_z^2\pi^2\hbar^2}{2mb^2}
\end{align*}
We observe what is formally known as degeneracy, where different combinations of of $n_x,n_y,n_z$, bring about the same total energy of the system.
\begin{align*}
	W_{n}=W_{1n}+W_{2n}+W_{3n}
\end{align*}

\begin{table}[t]
\begin{center}
\begin{tabular}{|l |l |l|l|} 
  \hline
 $(n_x,n_y,n_z)$ & Degeneracy& Energy ($\frac{\hbar^2}{2mb^2}$)  & Energy (Ev) \\
  \hline
  (1,1,1) & 1 & $3\pi^2$&10.8532\\ \hline
  (2,1,1),(1,2,1),(1,1,2) & 3 &$6\pi^2$&21.706 \\ \hline
  (2,2,1),(2,1,2),(1,2,2) & 3 &$9\pi^2$&32.5596 \\ \hline
  (3,1,1),(1,3,1),(1,1,3) & 3 &$11\pi^2$&39.795 \\ \hline
\end{tabular}
\caption{Table 2: Four lowest energies for cubic box}
\end{center}
\end{table}

Examining Table 1 and 2, we see that the cubic box case has slightly higher energies for each of the four lowest energy states; cubic energies are approximately $1$ to $2$ eV greater than the spherical energies. Also, the degeneracies are practically incomparable; the ground state and the first excited state share the same number of degeneracies in both the spherical and cubic case but the higher order excited states are incongruent. 
\end{document}

