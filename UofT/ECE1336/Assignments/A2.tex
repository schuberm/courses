\documentclass{article}

\usepackage{amsmath,graphicx,fullpage,parskip}
\newcommand{\unit}[1]{\ensuremath{\, \mathrm{#1}}}

\pagestyle{plain}
\numberwithin{equation}{section}

\begin{document}

\title{ECE1336 Semiconductor Physics: Assignment 2}
\author{Samuel Huberman (ID 999157923)}
\date{2011-09-22}
\maketitle

\section*{Problem 1}

The primitive vectors of a simple cubic lattice are:
\begin{align*}
\vec{a_1}&=a[1,0,0]\\
\vec{a_2}&=a[0,1,0]\\
\vec{a_3}&=a[0,0,1]\\
\end{align*}

The reciprical primitive vectors are defined through the relations:
\begin{align*}
	b_1 &= 2\pi \frac {\vec{a_2} \times \vec{a_3}}{\vec{a_1} \cdot (\vec{a_2} \times \vec{a_3})}
\\      b_2 &= 2\pi \frac {\vec{a_3} \times \vec{a_1}}{\vec{a_1} \cdot (\vec{a_2} \times \vec{a_3})}
\\      b_3 &= 2\pi \frac {\vec{a_1} \times \vec{a_2}}{\vec{a_1} \cdot (\vec{a_2} \times \vec{a_3})}
\end{align*}

Substituting the initial primitive vectors yields:
\\
\begin{align*}
\vec{b_1}&=\frac{2 \pi}{a}[1,0,0]\\
\vec{b_2}&=\frac{2 \pi}{a}[0,1,0]\\
\vec{b_3}&=\frac{2 \pi}{a}[0,0,1]\\
\end{align*}
\\
The reciprocal primitive vectors forms another simple cubic lattice with an edge length of $\frac{2\pi}{a}$.

\section*{Problem 2}
Bragg's law relates the lattice spacing of a crystal to the wavelength and the incident angle of reflection.

\begin{align*}
2d\sin \theta =n \lambda.
\end{align*}

The criteria of constructive inteference is that the differences in paths taken by the radiation must be an integer multiple of the wavelength.

For cubic systems, Bragg's law can be related to Miller indices of the Bravais lattices:

\begin{align*}
(\frac{\lambda}{2a})^2=\frac{\sin ^2 \theta}{h^2+k^2+l^2}
\end{align*}

To determine the planes that can be observed with a wavelenght of 3.1 Angstroms given a lattice spacing of 3.5 Angstroms with angle of incidence restriced from 0 to 90 degrees, we use any combination of values of h, k, and l that will obey the equality in the real plane.

\begin{bmatrix}
h & k & l\\
1 & 0 & 0\\
0 & 1 & 0\\
0 & 0 & 1\\
1 & 1 & 0\\
0 & 1 & 1\\
1 & 0 & 1\\
1 & 1 & 1\\
2 & 0 & 0\\
0 & 2 & 0\\
0 & 0 & 2\\
2 & 1 & 0\\
2 & 0 & 1\\
0 & 2 & 1\\
1 & 2 & 0\\
0 & 1 & 2\\
1 & 0 & 2\\
\end{bmatrix}

\section*{Problem 3}
\begin{itemize}
\item(a) The structure factor is given by:
\begin{align*}
F_K=\Sigma _j f_je^{-i \vec{K} \cdot\vec{r_j}}
\end{align*}
$j$ is a given atom in the unit cell, $f_j$ is the atomic form factor, $\vec{K}$ corresponds to a reciprocal lattice vector and $\vec{r_j}$ is the position of atom $j$.
The general form a reciprocal lattice vector for a body-centered cubic structure is
\begin{align*}
\vec{K}=\frac{2\pi}{a}[h,k,l]
\end{align*}
The BCC basis:
\begin{align*}
\vec{r_0}&=\frac{a}{2}[1,1,1]\\
\vec{r_1}&=[0,0,0]
\end{align*}
Substituting these relations into the structure factor equation, where $f$ is identical for every atom:
\begin{align*}
F_K&=f(e^{-i \frac{2\pi}{a}[h,k,l] \cdot [0,0,0]} + e^{-i \frac{2\pi}{a}[h,k,l] \cdot \frac{a}{2}[1,1,1]})
\\ &=f(1+(-1)^{h+k+l})
\end{align*}

\item(b)The general form a reciprocal lattice vector for a face-centered cubic structure is
\begin{align*}
\vec{K}=\frac{2}{a}[h,k,l]
\end{align*}
The FCC basis:
\begin{align*}\\
\vec{r_0}&=[0,0,0]\\
\vec{r_1}&=\frac{a}{2}[1,1,0]\\
\vec{r_2}&=\frac{a}{2}[0,1,1]\\
\vec{r_3}&=\frac{a}{2}[1,0,1]\\
\end{align*}
Substituting these relations into the structure factor equation, where $f$ is identical for every atom:
\begin{align*}
F_K &=f(e^{-i \frac{2}{a}[h,k,l] \cdot [0,0,0]} + e^{-i \frac{2}{a}[h,k,l] \cdot \frac{a}{2}[1,1,0]} + e^{-i \frac{2}{a}[h,k,l] \cdot \frac{a}{2}[0,1,1]}+ e^{-i\frac{2}{a}[h,k,l] \cdot \frac{a}{2}[1,0,1]})
\\ &=f(1+(-1)^{h+k}+(-1)^{k+l}+(-1)^{h+l})
\end{align*}
\item(c) The peak from $[1,1,1]$ corresponds to a FCC structure. BCC structures resolve peaks when the Miller indices $h+k+l$ equals an even number, as shown in expression for the geometric structure factor.


\end{itemize}
\end{document}
